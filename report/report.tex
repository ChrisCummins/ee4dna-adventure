\documentclass[12pt,a4paper]{article}

\usepackage{float}
\restylefloat{figure}
\usepackage{hyperref}
\usepackage{graphicx}
\usepackage{gensymb}
\usepackage[title]{appendix}
\usepackage[dotinlabels]{titletoc}
\usepackage[nottoc,numbib]{tocbibind}
\usepackage{mathtools}
\usepackage[margin=0.5in]{geometry}
\renewcommand{\thefootnote}{\arabic{footnote}}

\newcommand*\wrapletters[1]{\wr@pletters#1\@nil}
\def\wr@pletters#1#2\@nil{#1\allowbreak\if&#2&\else\wr@pletters#2\@nil\fi}

\usepackage{enumitem}
\setenumerate{itemsep=0pt}

% Add support for multi-page tables.
\usepackage{longtable}

\pagenumbering{arabic}

\title{EE4DNA Coursework 3}
\author{Chris Cummins}

\begin{document}
\maketitle

\section{Server Implementation}

The room server implements a simple maze in which a two dimensional
grid of procedurally generated rooms may be traversed. Two encode the
locational XY coordinates of the room into a single digit room number,
the following translation is used: $n_r = y_r \times x_{max} + x_r$,
where $x_{max}$ is the width of the maze. Since players always arrive
at room number 0, the locational coordinates of the rooms are centred
around the point $[0, 0]$, so for maze of height 20 and width 10, the
range of coordinates would be $[-10, -5]$ to $[10, 5]$.

The behaviour of rooms is configurable at run time, with the package
\texttt{adventure.actions} containing a set of classes which all
implement the \texttt{Action} interface, allowing a room to respond to
the user command. The idea is that classes may extend the base
\texttt{RoomImpl} class in order to provide extended functionality.
Each room object contains hash sets which track the players currently
in the room, and the set of actions that the room implements. The
actions implemented by the \texttt{DungeonRoom} class used in the maze
are shown in the help text:

\begin{verbatim}
[22:45:09] /description - describe the room
[22:45:09] /go <room|username> - move to a new room
[22:45:09] /say <message> - send a message to everyone
[22:45:09] /shout <message> - shout a message to everyone
[22:45:09] /show me a dragon - best command evah
[22:45:09] /how does star wars begin? - refresh your memory
[22:45:09] /tell me a story - regale a tale of great bravery
\end{verbatim}

Additionally, each \texttt{DungeonRoom} object has a description
associated with it which is selected from a set of generic RPG room
descriptions, contained in the \texttt{adventure.xml} configuration
file. The description command prints this room description, followed
by text to show whether the player is alone in the room, and the
directions in which the player can move:

\begin{verbatim}
[22:43:48] You open the door and a gout of flame rushes at your face. A wave  \
           of heat strikes you at the same time and light fills the hall. The \
           room beyond the door is ablaze! An inferno engulfs the place,      \
           clinging to bare rock and burning without fuel. You are alone.     \
           From here, you can go north, east, south, west.
\end{verbatim}

\newpage
\section{Lazy Instantiation Process}

Rooms are instantiated lazily by the \texttt{Maze} object using a
\texttt{DungeonRoomFactory}. Upon initialisation, no rooms exist, and
rooms are created only as required. Each room requires a
\texttt{CBRoomServer} object in its constructor, which is my solution
for resolving the circular dependency between the Room Server and the
Room Server callback provided by the game server upon registration. By
postponing room instantiation until a room is requested, there is no
opportunity for race conditions, since a room can only be requested by
a player once the room server has been registered, at which point it
has access to the room server callback object.

Upon calling the \texttt{find\_room()} method, the room server
validates the requested room number by ensuring that value is within
the legal range (defined by the range of locational coordinates). If
not, a \texttt{room\_not\_found} exception is thrown. If the room
number is legal, then the map of instantiated room objects is
inspected to see if it contains a tuple with that room number as a the
key. If so, the value of the key is returned. If not, a room object is
instantiated using the \texttt{DungeonRoomFactory} object, and this
new room is inserted into the map and returned to the caller. By
instantiating rooms lazily in this fashion, the memory footprint and
initialisation times of the room server are minimised. Inspecting the
server log as a player moves west within a newly instantiated maze
shows this pattern of room generation occurring:

\begin{verbatim}
Sun Mar 02 22:43:44 GMT 2014 OK:      Generating room -1
Sun Mar 02 22:43:44 GMT 2014 OK:      ping() invoked on Room 'cummince'
Sun Mar 02 22:43:44 GMT 2014 OK:      player_left(cummince)
Sun Mar 02 22:43:44 GMT 2014 OK:      player_entered(cummince)
Sun Mar 02 22:43:45 GMT 2014 OK:      Generating room -2
Sun Mar 02 22:43:45 GMT 2014 OK:      ping() invoked on Room 'cummince'
Sun Mar 02 22:43:45 GMT 2014 OK:      player_left(cummince)
Sun Mar 02 22:43:45 GMT 2014 OK:      player_entered(cummince)
Sun Mar 02 22:43:45 GMT 2014 OK:      Generating room -3
Sun Mar 02 22:43:45 GMT 2014 OK:      ping() invoked on Room 'cummince'
Sun Mar 02 22:43:45 GMT 2014 OK:      player_left(cummince)
Sun Mar 02 22:43:45 GMT 2014 OK:      player_entered(cummince)
Sun Mar 02 22:43:45 GMT 2014 OK:      Generating room -4
Sun Mar 02 22:43:45 GMT 2014 OK:      ping() invoked on Room 'cummince'
Sun Mar 02 22:43:45 GMT 2014 OK:      player_left(cummince)
Sun Mar 02 22:43:45 GMT 2014 OK:      player_entered(cummince)
Sun Mar 02 22:43:46 GMT 2014 OK:      Generating room -5
Sun Mar 02 22:43:46 GMT 2014 OK:      ping() invoked on Room 'cummince'
Sun Mar 02 22:43:46 GMT 2014 OK:      player_left(cummince)
Sun Mar 02 22:43:46 GMT 2014 OK:      player_entered(cummince)
\end{verbatim}

\newpage
\section{Automated Testing}

The \texttt{test/run} script was written to provide black-box testing
of the server, running through the majority of possible error states
in an automated fashion. The script performs a sequence of tests using
different configurations, and for each, the output is displayed. A
full copy of the automated test output can be found in
reports/test.txt. Further testing could be performed at an API level
using JUnit, although the number of edge cases that can be tested is
limited by the fact that we are relying on the external game
server. If we were to test for run time errors such as losing
connection with the remote server, then we would need to implement a
mock games server which we could host locally.

Improved white-box testing could be implemented by reverse engineering
the GWT client application used to control players. An inspection of
the HTTP headers sent during a game session show repeated POST
requests being made to the Java RPC service, with game state data
being transmitted to the client as JSON:

\begin{verbatim}
POST /applications/gameserver/rpc HTTP/1.1
Host: aries.aston.ac.uk:443
Accept: */*
Accept-Encoding: gzip,deflate,sdch
Accept-Language: en-US,en;q=0.8,en-GB;q=0.6
Content-Type: text/x-gwt-rpc; charset=UTF-8
Cookie: <snip>
Origin: https://aries.aston.ac.uk
Referer: https://aries.aston.ac.uk/applications/gameserver.html
User-Agent: <snip>
X-GWT-Module-Base: https://aries.aston.ac.uk/applications/gameserver/
X-GWT-Permutation: 6AB91D4FBF4382B344FE89D59DC47F4D

HTTP/1.1 200 OK
Connection: close
Content-Disposition: attachment
Content-Length: 115
Content-Type: application/json;charset=utf-8
Date: Sun, 02 Mar 2014 02:40:24 GMT
\end{verbatim}

However, it would be a substantial challenge to reverse-engineer the
client side JavaScript, so manual testing provides much more value for
effort. Alternatively, a browser automation test harness could be used
such as selenium, but this brings with it it's own requirements in
terms of set up time, and wouldn't offer any level of understanding of
the application, only the ability to repeatedly perform a set of
scripted or recorded commands.

\end{document}
